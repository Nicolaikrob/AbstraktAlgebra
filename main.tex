\documentclass{article}
\usepackage[utf8]{inputenc}
\usepackage[danish,english]{babel}
\usepackage{amsmath, amsfonts, amsthm, amssymb, tikz, graphicx}
\usepackage[margin=2cm]{geometry}
\usepackage[ddmmyy]{datetime}

\newcommand{\R}{\mathbb{R}}
\newcommand{\N}{\mathbb{N}}
\newcommand{\Z}{\mathbb{Z}}
\newcommand{\Q}{\mathbb{Q}}
\newcommand{\F}{\mathbb{F}}

\newcommand{\inv}{^{-1}}
\newcommand{\imp}{\Rightarrow}
\newcommand{\biimp}{\Leftrightarrow}
\newcommand{\vimp}{\Downarrow}
\newcommand{\limm}[2]{\lim_{#1 \rightarrow #2}}
\newcommand{\modstrid}{\lightning}
\newcommand{\too}{\rightarrow}
\newcommand{\from}{\leftarrow}

\newcommand{\vecdd}[2]{\begin{pmatrix} #1 \\ #2 \end{pmatrix}}
\newcommand{\vecddd}[3]{\begin{pmatrix} #1 \\ #2 \\ #3 \end{pmatrix}}

\newcommand{\s}[2]{\begin{#1} #2 \end{#2}}
\newcommand{\ud}[1]{}

\newtheorem{opg}{Opgave}
\newtheorem{lem}{Lemma}
\newtheorem{kor}{Korolar}
\newtheorem{setn}{Sætning}
\newtheorem{defi}{Definition}
\newtheorem{bem}{Bemærkning}
\newtheorem{nota}{Notation}

\renewcommand*{\proofname}{Bevis}

\newcommand{\complexAxis}[1]{
	\draw [->] (-#1-1,0) -- (#1+1,0);
	\draw [->] (0,-#1-1) -- (0,#1+1);
	\foreach \n in {-#1,...,-1,1,2,...,#1}{
		\draw (\n,-3pt) -- (\n,3pt)   node [above] {$\n$};
		\draw (-3pt,\n) -- (3pt,\n)   node [right] {$\n i$};
	}
}


\title{Primitivt Algebra Brevkursus}
\author{Malte Kildelund Rosenkilde}

\begin{document}

	\maketitle
	\newpage

	\section*{Disclaimer}
		Jeg kommer nok til at lave en masse fejl så tag ikke alt som værende helt
		korrekt. Så stil endeligt spørgsmål hvis der er noget der ser forkert ud
		eller ikke giver mening.
		Stave fejl er nok også noget der kommer til at være meget af.
		Alt jeg ved er fra bogen
		Abstract Algebra, 3rd Edition af David S. Dummit og Richard M. Foote
		så læs i den hvis der er brug for bedre kilder.
	\section*{Grupper 21/2}
		Det helt basele i abstract algebra er grupper hvilket er en struktur der
		ses overalt i matematikken.
		\subsection*{Teori}
		\begin{defi}
			Lad $G$ være en mængde, da er en function $*: G \times G \too G$
			en binær operation.

			Som notation skrives $a*b$ istedet for $*(a,b)$.

			En binær operation $*$ kaldes ascosiativ hvis
			$\forall a,b,c \in G: a*(b*c) = (a*b)*c$.

			En binær operation $*$ kaldes kommutativ hvis
			$\forall a,b \in G: a*b = b*a$.
		\end{defi}
		\begin{defi}
			En tupel $(G,*)$ med en mængde $G$ og en binær operation $*$ kaldes
			en gruppe hvis:

			(1) $*$ er ascosiativ.

			(2) Der eksistere et element $e \in G$ så $\forall a \in G: a*e=e*a=a$
			kaldet det neutrale element.

			(3) For alle elementer $a \in G$ eksistere $a\inv \in G$ så
			$a*a\inv = a\inv*a = e$ kaldet det inverse element til $a$.

			En gruppe kaldes abelsk hvis $*$ er kommutativ.

			Ofte kalder betegner man $G$ for gruppen istedet for $(G,*)$ og
			da er operationen implicit.
		\end{defi}
		Som notation bruges der ofte $\cdot$ som operation istedet for $*$ og
		$a\cdot b$ bliver ofte skrevet $ab$ istedet. Det neutrale element bliver så
		betegnet $1$. Dog er det normalt at bruge
		$+$ for opreationen i abelsek grupper og at bruge $-a$ istedet for $a\inv$.
		Dog er $-$ ikke en operation her men der skrives stadig $a-b$ istedet for $a+-b$.
		\subsection*{Sætninger}
		\begin{setn}
			Neutrale elementer er unikke. Altså givet en gruppe $(G,*)$ og to elementer
			$e_1, e_2 \in G$ hvor $\forall a \in G: e_1*a=a*e_1=a$ og $e_2*a=a*e_2=a$
			da er $e_1 = e_2$.
		\end{setn}
		\begin{proof}
			$$e_1 = e_1*e_2 = e_2$$
		\end{proof}
		\subsection*{Vis selv}
		\begin{setn}
			Invers elementer er unikke. Altså givet en gruppe $(G,*)$ og tre element
			$a,a_1\inv,a_2\inv\in G$ hvor $a*a_1\inv=a_1\inv*a=e$ og $a*a_2\inv=a_2\inv*a=e$
			da er $a_1\inv = a_2\inv$.
		\end{setn}
		\begin{setn}
			Givet en gruppe $(G,*)$ og et element $a \in G$ da er $(a\inv)\inv = a$.
		\end{setn}
		\begin{setn}
			Givet en gruppe $(G,*)$ og to elementer $a,b \in G$ da er $(a*b)\inv=b\inv*a\inv$.
		\end{setn}
	\newpage
	\section*{Homomorphier 22/2}
		En vigtig del af abstract algebra er at se på relationer mellem
		forskellige strukture hvilket gøres ved hjælp af homomorphier og isomorphier.
		\subsection*{Definitioner}
		\begin{defi}
			Lad $(G,*)$ og $(G,\diamond)$ være to grupper og $\varphi: G \too H$ være
			en function. Da kaldes $\varphi$ en gruppe homomorphi hvis
			$$\forall a, b \in G: \varphi(a*b) = \varphi(a) \diamond \varphi(b)$$

			En bijektiv homomorphi kaldes en isomorphi.

			To grupper $G$ og $H$ kaldes isomorphe hvis der eksistere en isomorphi
			mellem dem. Dette skrives $G \cong H$.

			En isomorphi $\varphi: G \too G$ mellem en gruppe $G$ og den selv kaldes for en
			automorphi på $G$.
		\end{defi}
		\begin{defi}
			Lad $G$ og $H$ være to grupper med identiterter $e_G$ og $e_H$
			og $\varphi: G \too H$ være en homomorphi.
			Da betegner kernen af $\varphi$ mængde af elementer som bliver
			afbilledet til $e_H$.

			$$\ker(\varphi) = \{g \in G | \varphi(g) = e_H\}$$
		\end{defi}
		\subsection*{Sætninger}
		\begin{setn}
			For to grupper $(G,*)$ og $(H, \diamond)$ med neutrale elementer $e_G$ og $e_H$
			og en homomorphi $\varphi: G \too H$ da er $\varphi(e_1) = e_2$.
		\end{setn}
		\begin{proof}
			$$\varphi(e_G) = \varphi(e_G)\diamond e_H
			= \varphi(e_G)\diamond\varphi(e_G)\diamond\varphi(e_G)\inv
			= \varphi(e_G*e_G)\diamond\varphi(e_G)\inv
			= \varphi(e_G)\diamond\varphi(e_G)\inv = e_H$$
		\end{proof}
		\subsection*{Vis selv}
		\begin{setn}
			For to grupper $G$ og $H$, en homomorphi $\varphi: G \too H$
			og et element $a \in G$ da er $\varphi(a)\inv = \varphi(a\inv)$.
		\end{setn}
		\begin{setn}
			For to grupper $G$ og $H$ eksistere der altid en homomoprhi mellem dem.
		\end{setn}
		At to grupper er isomorphe betyder at deres struktur er meget ens og isomorphier
		fungere næsten som en ekvilens relation hvilket ses i følgene opgave.
		\begin{setn}
			Isomorphier opfylder kravende for en ekvivilens relation:

			$\cong$ er refleksiv altså $G \cong G$ for alle grupper $G$.

			$\cong$ er symmetrisk altså $G \cong H \biimp H \cong G$ for
			alle grupper $G$ og $H$.

			$\cong$ er transitiv altså $G \cong H \wedge H \cong K \imp G \cong K$
			for alle grupper $G$, $H$ og $K$.

		\end{setn}
		Årsagen til at det ikke er en ekvivilens relations skyldes at mængde lærer ikke
		kan lide at konstruere en mængde af alle grupper og der dermed ikke er en mængde
		ekvivilens relationen kan være over.
		\begin{setn}[$\star$] \label{KerInj}
			Lad $G$ og $H$ være to grupper med identiterter $e_G$ og $e_H$
			og $\varphi: G \too H$ være en homomorphi.

			Da er $\varphi$ injektiv hvis og kun hvis $\ker(\varphi) = \{e_G\}$.
		\end{setn}
		Denne sætninger ret relevant så jeg skriver beviset i næste opdatering, det er
		dog stadig en ret god øvelse at vise.
	\newpage
	\section*{Undergrupper 23/2}
		\subsection*{Opsamling}
		Her er beviset for sætning \ref{KerInj}.
		\begin{proof}
			Lad $a,b \in G$.
			Hvis $\ker(\varphi) = {e_G}$ da ses det at
			$$\varphi(a) = \varphi(b) \imp \varphi(a*b\inv)=\varphi(a)\varphi(b)\inv=e_H
			\imp a*b\inv \in \ker(\varphi) \imp a*b\inv = e_G \imp a = b$$
			Hvis $\ker(\varphi)$ ikke er triviel er functionen åbenlyst ikke injektiv.
		\end{proof}
		\subsection*{Definitioner}
		Fra nu af vil noten skifte over til multiplikativ notation så operationer
		er underforstået i forhold til hvor de sker og der bliver brugt. $1$ bliver
		også brugt som enhed.
		$a\cdot b$ eller bare $ab$.
		\begin{defi}
			Lad $G$ være en gruppe og $H \ne Ø \subseteq G$ være en delmængde.
			Da er $H$ en undergruppe af $G$ noteret $H \le G$ hvis

			(1) $x \in H \imp x\inv \in H$

			(2) $x,y \in H \imp xy \in H$
		\end{defi}
		\begin{nota}
			Lad $G$ være en gruppe, $H \le G$ og $g \in G$. Da er der følgende notation

			$gH = \{gh | \forall h \in H\}$ Kaldet en venstrsideklasse.

			$Hg = \{hg | \forall h \in H\}$ Kaldet en højresideklasse.

			$gHg\inv = \{ghg\inv | \forall h \in H\}$
			Kaldet $H$ konjugeret med $g$ ligesom $ghg\inv$ er $h$ konjugeret med $g$.
		\end{nota}
		\subsection*{Sætninger}
		\begin{setn} \label{SidKlaPar}
			Lad $G$ være en gruppe og $H \le G$. For elementer $a, b \in G$ da er
			$aH = bH$ eller $aH\cap bH = Ø$.
		\end{setn}
		\begin{proof}
			Antag at der eksister $c \in aH\cap bH$. Da ligger $c$ både i $aH$ og i $bH$ så
			der må eksistere $h_1, h_2$ så $c = ah_1$ og $c = bh_2$. Da ses det at
			$$ah_1 = bh_2 \imp a=bh_2h_1\inv$$
			Lad nu $d$ være et element
			i $aH$. Da ses det at
			$$d = ah_3=bh_2h_1\inv h_3$$
			Men da $h_1$, $h_2$ og $h_3$ ligger i $H$ må $h_2h_1\inv h_3$ ligge i $H$
			da $H$ er en undegruppe. Altså må $d$ ligge i $bH$ og dermed er $aH \subseteq bH$.
			Det ses symmetrisk at $bH \subseteq aH$ hvilket medføre $aH = bH$.
		\end{proof}
		\subsection*{Vis selv}
		\begin{setn}
			Lad $G$ være en gruppe og $H \le G$. Da gælder det at $1 \in H$.
		\end{setn}
		\begin{setn}
			Lad $G$ og $H$ være to grupper og $\varphi: G \too H$ være en homomorphi.
			Da er både $\ker(\varphi)$ og $\varphi(G)$ undergrupper af $H$.
			
			(Note: $\varphi(G)$ er billedet af $\varphi$ ofte skrevet $\text{im}(\varphi)$.)
		\end{setn}
		\begin{setn} \label{SidKlaStø}
			Lad $G$ være en gruppe, $H \le G$ og $a,b \in G$. Da er $|aH| = |bH|$.
			Hvilket er ekvivilent med at der eksistere en bijektion mellem $|aH|$ og $|bH|$.
		\end{setn}
		\begin{setn}[Lagrange $\star$] \label{Lagrange}
			Lad $G$ være en endelig gruppe og $H \le G$. Da gælder det at
			$$|H| \,\Big|\, |G|$$
			Læses $|H|$ deler $|G|$.
		\end{setn}
		(Hint: Benyt sætning \ref{SidKlaStø} og \ref{SidKlaPar}.)
	\newpage
	\section*{Normale undergrupper 24/2}
		Idag bliver lidt kortere.
		\subsection*{Opsamling}
		Her er beviset for \ref{Lagrange}.
		\begin{proof}
			For et givent element $g \in G$ må $g \in gH$ da $1 \in H$.
			Fra sætning \ref{SidKlaPar} ses det så at $H$ sideklasserne
			er en partition af $G$. Lad $K$ betgne mægnden af $H$ sideklasser da må
			$$|G| = \sum_{S \in K} |S|$$
			Fra \ref{SidKlaStø} fåes det at alle $H$ sideklasser har samme størelser og
			da $H$ er en $H$ sideklasse har de alle størelse $|H|$. Det ses så at
			$$|G| = \sum_{S \in K} |S| = \sum_{S \in K} |H| = |H|\cdot |K|$$
			Da $G$ er endelig må både $H$ og $K$ være endelige og $|G|$, $|H|$ og $|K|$
			må da være hele tal og derfor må $|H| \,\Big|\, |G|$.
		\end{proof}
		\subsection*{Definitioner}
		Det giver nu mening at tale om mængden af sideklasser.
		\begin{defi}
			Lad $G$ være en gruppe og $H \le G$. Da betegner $|G:H|$
			antallet af $H$ sideklasser. $|G:H|$ kan godt være uendelig.
		\end{defi}
		\begin{defi}
			Lad $G$ være en gruppe og $H \le G$. Da er $H$ en normal undergruppe hvis
			$$\forall g \in G: gHg\inv = H$$
			Hvilket skrives $H \unlhd G$.
		\end{defi}
		\subsection*{Vis selv}
		\begin{setn}
			Lad $G$ og $H$ være grupper og $\varphi: G \too H$ være en homomorphi.
			Da er $\ker(\varphi) \unlhd H$.
		\end{setn}
	\newpage
	\section*{Kvotientgrupper 25/2}
		\subsection*{Definitioner}
		\begin{defi}
			Lad $G$ være en gruppe og $H \unlhd G$ da betegner $G/H$ gruppen
			af $H$ sideklasserne hvor for to side klasser $aH$ og $bH$ hvor $a$ og $b$
			er to vilkårlige representanter er $aH\cdot bH = abH$. Denne gruppe er kaldet
			en kvotientgruppe.
			Homomorphien $\pi: G \too G/H$ defineret ved $\pi(g) = gH$ er kaldet
			den kanoniske afbildning.
		\end{defi}
		\begin{defi}
			Lad $G$ være en gruppe og $H \le G$. Da er der følgende undergrupper:
			$$N_G(H) = \{g \in G| gHg\inv = H\}$$
			Kaldet normalisatoren til $H$.
			$$C_G(H) = \{g \in G| \forall h \in H: gh = hg\}$$
			Kaldet centralisatoren til $H$.
			$Z(G) = C_G(G)$ kaldet centeret i $G$.
		\end{defi}
		\subsection*{Sætninger}
		\begin{setn}
			Kvotientgruppe operationen er veldefineret.
		\end{setn}
		\begin{proof}
			Lad $a,a',b,b' \in G$ så $aH = a'H$ og $bH = b'H$. Da eksistere $h_1, h_2 \in H$
			så $a' = ah_1$ og $b' = bh_2$. Betragt nu
			$$a'b'H = ah_1bh_2H = abb\inv h_1bH$$
			Men da $H$ er en normal under gruppe ligger $b\inv h_1 b = h_3 \in H$.
			$$a'b'H = abb\inv h_1bH = abh_3H = abH$$
			Ergo er valget af representatner for gruppe operationen ligegyldig og operationen
			er dermed veldefineret.
		\end{proof}
		\begin{setn}
			Kvotientgrupper er grupper.
		\end{setn}
		\begin{proof}
			Det ses at der både er inverser, et neutralt element og at operationen er asosiativ:

			$1H\cdot aH = aH$,

			$aH\cdot a\inv H = 1H$,

			$aH(bH\cdot cH) = aH \cdot bcH = abcH = abH \cdot cH = (aH\cdot bH)cH$.
		\end{proof}
		\begin{setn}
			Lad $G$ være en gruppe og $H \le G$. Da gælder det at 
			$$H \biimp a\inv b \in H \biimp aH = bH$$
		\end{setn}
		\begin{proof}
			$$aH = bH \biimp \exists h \in H: b = ah \biimp \exists h \in H: a\inv b = h \biimp
			a\inv b \in H$$
			Første biimplikation fåes fra sætning \ref{SidKlaPar}.
		\end{proof}
		\subsection*{Vis selv}
		\begin{setn}
			Lad $G$ væren en gruppe og $H \le G$, da er både $C_G(H)$ og $N_G(H)$ undergrupper
			af $G$.
		\end{setn}
		\begin{setn}
			Lad $G$ være en gruppe og $H \unlhd G$ da er
			den kanoniske afbildning en homomorphi.
		\end{setn}
		\begin{opg}
			Vis at $(3\Z,+)$ er en gruppe og bestem $\Z/3\Z$.

			Bestem $\Z/n\Z$ for et naturligt tal $n$.

			(Bemærkning: $n\Z = \{n\cdot a| \forall a \in \Z\}$.)
		\end{opg}
		\begin{setn}[Første isomorphi sætning $\star$] \label{Iso1}
			Lad $G$ og $H$ være grupper og lad $\varphi: G \too H$ være en gruppe homomorphi.
			Da er $G/\ker(\varphi) \cong \varphi(G)$.
		\end{setn}
	\newpage
	\section*{Isomorphi sætninger del 1 26/2}
		\subsection*{Opsamling}
		Her er beviset for sætning \ref{Iso1}.
		\begin{proof}
			Betragt functionen $\pi: G/\ker(\varphi) \too \phi(G)$ defineret ved
			$$\pi(g\ker(\varphi)) = \varphi(g)$$
			Først ses det at $\pi$ er veldefineret. Så antag $g\ker(\varphi) = g'\ker(\varphi)$.
			Der eksistere $k \in \ker(\varphi)$ så
			$$\pi(g\ker(\varphi)) = \varphi(g) = \varphi(g'k) = \varphi(g')\varphi(k)=\varphi(g') =
			\pi(g'\ker(\varphi))$$
			Det ses nu at $\pi$ er en homomorphi:
			$$\pi(a\ker(\varphi)b\ker(\varphi)) = \pi(ab\ker(\varphi)) = \varphi(ab) =
			\varphi(a)\varphi(b)=\pi(a\ker(\varphi))\pi(b\ker(\varphi))$$
			Det ses let at $\pi$ er surjektiv da hvis $h \in \varphi(G)$ eksistere $g$
			så $\varphi(g) = h$ og da er $\pi(g\ker(\varphi)) = \varphi(g) = h$.
			Det ses også at $\pi$ er injektiv da
			$$\pi(g\ker(\varphi)) = 1 \imp \varphi(g) = 1 \imp g \in \ker(\varphi) \imp
			g\ker(\varphi) = 1\ker(\varphi)$$
			Da er kernen af $\pi$ triviel og fra sætning \ref{KerInj} må $\pi$ være injektiv.
			Ergo er $\pi$ en isomorphi.
		\end{proof}
		\subsection*{Definitioner}
		\begin{defi}
			Lad $G$ være en gruppe og $A$ og $B$ være undergrupper af $G$. Da betegner
			$$AB = \{ab | \forall a \in A \forall b \in B\}$$
		\end{defi}
		\subsection*{Sætninger}
		\begin{setn}
			Lad $G$ være en gruppe og $A \subseteq N_G(B)$ og $B$ være undergrupper af $G$.
			Da er $AB \le G$.
		\end{setn}
		\begin{proof}
			Betragt $a \in A$ og $b \in B$ da $A \subseteq N_G(B)$ er $ab\inv a\inv = b' \in B$ og
			$$(ab)\inv = b\inv a\inv = a\inv a b \inv a \inv = a \inv b'$$
			Da $a\inv \in A$ og $b' \in B$ ses det så at $(ab)\inv \in AB$. Betragt nu
			$a_1, a_2 \in A$ og $b_1, b_2 \ in B$ da må $a_2\inv b_1 a_2 = b' \in B$ det ses så at
			$$a_1b_1a_2b_2 = a_1a_2a_2\inv b_1a_2b_2 = a_1a_2b'b_2$$
			ergo ligger $(a_1b_1)(a_2b_2)$ i $AB$ og $AB$ må være en undergruppe.
		\end{proof}
		\begin{setn}[Den anden isomorphi sætning]
			Lad $G$ være en gruppe og $A \subset N_G(B)$ og $B$ være undergrupper af $G$.
			Da er $B \unlhd AB$, $a\cap B \unlhd A$ og $AB/B \cong A/A\cap B$.
		\end{setn}
		\begin{proof}
			Beviset for at $B \unlhd AB$ og $A\cap B \unlhd A$ undlades til læseren(Dagens opgaver).
			Bestem $\varphi: A \too AB/B$ ved $\varphi(a) = aB$. Da ses det at $\varphi$ er en
			homomorphi:
			$$\varphi(aa') = aa'B = aBa'B = \varphi(a)\varphi(a')$$
			Vi bestemmer nu $\ker(\varphi)$ så betragt $a \in A$ så $\varphi(a) = 1B$
			Da må $aB = \varphi(a) = 1B$ ergo må $a \in B$ og kernen er dermed $A\cap B$.
			Det ses også at $\varphi$ er surjectiv da $aB = \varphi(a')$ for enhver representant
			$a' \in aB$. Fra sætning \ref{Iso1} fåes det så at
			$$A/A\cap B = A/\ker(\varphi) \cong \varphi(A) = AB/B$$
		\end{proof}
	\newpage
	\section*{Isomorphi sætninger del 2 27/2}
		\subsection*{Vis selv}
		\begin{setn}
			Lad $G$ være en gruppe, $H \unlhd G$, $K \unlhd G$ og $H \le K$.

			Da er $K/H \unlhd G/H$ og
		\end{setn}
		\subsection*{Sætninger}
		\begin{setn}[Tredje isomorphi sætning] \label{Iso3}
			Lad $G$ være en gruppe, $H \unlhd G$, $K \unlhd G$ og $H \le K$.

			Da er $(G/H)/(K/H) \cong G/K$
		\end{setn}
		\begin{proof}
			Definer $\varphi: G/H \too G/K$ ved $\varphi(gH) = gK$. Først ses
			det at $\varphi$ er veldefineret. Lad $gH = g'H$ da eksister $h$ så
			$g' = gh$. Da $H \le K$ ses det at
			$$\varphi(g'H) = g'K = ghK = gK$$
			Vi bestemmer nu $\ker(\varphi)$.
			\begin{align*}
				\ker(\varphi) &= \{gH \in G/H| \varphi(gH) = 1K\} \\
				&= \{gH \in G/H| gK = 1K\} \\
				&= \{gH \in G/H| g \in K\} \\
				&= K/H
			\end{align*}
			Det ses let at $\varphi$ er surjektiv og fra sætning \ref{Iso1} ses det så at
			$$(G/H)/(K/H) = (G/K)/\ker(\varphi) \cong = \varphi(G/K) = G/K$$
		\end{proof}
		Af isomorphi sætninger er det den første der er den stærkeste hvilket de to andre viser
		da beviserne falder ud ved at betragte den mest trivielle homomorphi og der efter bruge
		første isomorphi sætning.

		Næste gang begynder vi på det sidste store resultat i gruppe teori vi vil se på før vi
		går videre til ringe og legmer.
	\ud{
	\newpage
	\section*{Start på Sylow}
		\subsection*{Opsamling}
		\subsection*{Definitioner}
		\subsection*{Sætninger}
		\subsection*{Vis selv}
	\section*{Til senere}
		\begin{defi}
			En tupel $(R,+,\cdot)$ med en mængde $R$ og binære operationer $+, \cdot$
			kaldes en ring hvis:

			(1) $(R,+)$ er en gruppe hvor neutral elementet kaldes $0$.

			(2) $\cdot$ er en acosiativ operation.

			(3) $\cdot$ distribuere over $+$ alstå $\forall a,b,c \in R: a*(b+c)=a*b+a*c$
			og $(a+b)*c = a*c+b*c$.
		\end{defi}
	\section*{Til senere}
	}

\end{document}
