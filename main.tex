\documentclass{article}
\usepackage[utf8]{inputenc}
\usepackage[danish,english]{babel}
\usepackage{amsmath, amsfonts, amsthm, amssymb, tikz, graphicx}
\usepackage[margin=2cm]{geometry}
\usepackage[ddmmyy]{datetime}

\newcommand{\R}{\mathbb{R}}
\newcommand{\N}{\mathbb{N}}
\newcommand{\Z}{\mathbb{Z}}
\newcommand{\Q}{\mathbb{Q}}
\newcommand{\F}{\mathbb{F}}

\newcommand{\inv}{^{-1}}
\newcommand{\imp}{\Rightarrow}
\newcommand{\biimp}{\Leftrightarrow}
\newcommand{\vimp}{\Downarrow}
\newcommand{\limm}[2]{\lim_{#1 \rightarrow #2}}
\newcommand{\modstrid}{\lightning}
\newcommand{\too}{\rightarrow}
\newcommand{\from}{\leftarrow}

\newcommand{\vecdd}[2]{\begin{pmatrix} #1 \\ #2 \end{pmatrix}}
\newcommand{\vecddd}[3]{\begin{pmatrix} #1 \\ #2 \\ #3 \end{pmatrix}}

\newcommand{\s}[2]{\begin{#1} #2 \end{#2}}
\newcommand{\ud}[1]{}

\newtheorem{opg}{Opgave}
\newtheorem{setn}{Sætning}
\newtheorem{lem}[setn]{Lemma}
\newtheorem{koro}[setn]{Korolar}
\newtheorem{defi}{Definition}
\newtheorem{bem}{Bemærkning}
\newtheorem{nota}{Notation}

\renewcommand*{\proofname}{Bevis}

\newcommand{\complexAxis}[1]{
	\draw [->] (-#1-1,0) -- (#1+1,0);
	\draw [->] (0,-#1-1) -- (0,#1+1);
	\foreach \n in {-#1,...,-1,1,2,...,#1}{
		\draw (\n,-3pt) -- (\n,3pt)   node [above] {$\n$};
		\draw (-3pt,\n) -- (3pt,\n)   node [right] {$\n i$};
	}
}


\title{Primitivt Algebra Brevkursus}
\author{Malte Kildelund Rosenkilde}

\begin{document}

	\maketitle
	\newpage

	\section*{Disclaimer}
		Jeg kommer nok til at lave en masse fejl så tag ikke alt som værende helt
		korrekt. Så stil endeligt spørgsmål hvis der er noget der ser forkert ud
		eller ikke giver mening.
		Stave fejl er nok også noget der kommer til at være meget af.
		Alt jeg ved er fra bogen
		Abstract Algebra, 3rd Edition af David S. Dummit og Richard M. Foote
		så læs i den hvis der er brug for bedre kilder.
	\section*{Grupper 21/2}
		Det helt basele i abstract algebra er grupper hvilket er en struktur der
		ses overalt i matematikken.
		\subsection*{Teori}
		\begin{defi}
			Lad $G$ være en mængde, da er en function $*: G \times G \too G$
			en binær operation.

			Som notation skrives $a*b$ istedet for $*(a,b)$.

			En binær operation $*$ kaldes ascosiativ hvis
			$\forall a,b,c \in G: a*(b*c) = (a*b)*c$.

			En binær operation $*$ kaldes kommutativ hvis
			$\forall a,b \in G: a*b = b*a$.
		\end{defi}
		\begin{defi}
			En tupel $(G,*)$ med en mængde $G$ og en binær operation $*$ kaldes
			en gruppe hvis:

			(1) $*$ er ascosiativ.

			(2) Der eksisterer et element $e \in G$ så $\forall a \in G: a*e=e*a=a$
			kaldet det neutrale element.

			(3) For alle elementer $a \in G$ eksisterer $a\inv \in G$ så
			$a*a\inv = a\inv*a = e$ kaldet det inverse element til $a$.

			En gruppe kaldes abelsk hvis $*$ er kommutativ.

			Ofte betegner man gruppen $G$ istedet for $(G,*)$ og
			da er operationen implicit.
		\end{defi}
		Som notation bruges der ofte $\cdot$ som operation istedet for $*$ og
		$a\cdot b$ bliver ofte skrevet $ab$ istedet. Det neutrale element bliver så
		betegnet $1$. Dog er det normalt at bruge
		$+$ for opreationen i abelske grupper og at bruge $-a$ istedet for $a\inv$.
		Dog er $-$ ikke en operation her men der skrives stadig $a-b$ istedet for $a+-b$.
		\subsection*{Sætninger}
		\begin{setn}
			Neutrale elementer er unikke. Altså givet en gruppe $(G,*)$ og to elementer
			$e_1, e_2 \in G$ hvor $\forall a \in G: e_1*a=a*e_1=a$ og $e_2*a=a*e_2=a$
			da er $e_1 = e_2$.
		\end{setn}
		\begin{proof}
			$$e_1 = e_1*e_2 = e_2$$
		\end{proof}
		\subsection*{Vis selv}
		\begin{setn}
			Invers elementer er unikke. Altså givet en gruppe $(G,*)$ og tre elementer
			$a,a_1\inv,a_2\inv\in G$ hvor $a*a_1\inv=a_1\inv*a=e$ og $a*a_2\inv=a_2\inv*a=e$
			da er $a_1\inv = a_2\inv$.
		\end{setn}
		\begin{setn}
			Givet en gruppe $(G,*)$ og et element $a \in G$ da er $(a\inv)\inv = a$.
		\end{setn}
		\begin{setn}
			Givet en gruppe $(G,*)$ og to elementer $a,b \in G$ da er $(a*b)\inv=b\inv*a\inv$.
		\end{setn}
	\newpage
	\section*{Homomorfier 22/2}
		En vigtig del af abstrakt algebra er at se på relationer mellem
		forskellige strukturer, hvilket gøres ved hjælp af homomorfier og isomorfier.
		\subsection*{Definitioner}
		\begin{defi}
			Lad $(G,*)$ og $(G,\diamond)$ være to grupper og $\varphi: G \too H$ være
			en funktion. Da kaldes $\varphi$ en gruppe homomorfi hvis
			$$\forall a, b \in G: \varphi(a*b) = \varphi(a) \diamond \varphi(b)$$

			En bijektiv homomorfi kaldes en isomorfi.

			To grupper $G$ og $H$ kaldes isomorfe hvis der eksisterer en isomorfi
			mellem dem. Dette skrives $G \cong H$.

			En isomorfi $\varphi: G \too G$ mellem en gruppe $G$ og den selv kaldes for en
			automorfi på $G$.
		\end{defi}
		\begin{defi}
			Lad $G$ og $H$ være to grupper med identiteter $e_G$ og $e_H$
			og $\varphi: G \too H$ være en homomorfi.
			Da betegner kernen af $\varphi$ mængden af elementer som bliver
			afbilledet til $e_H$.

			$$\ker(\varphi) = \{g \in G | \varphi(g) = e_H\}$$
		\end{defi}
		\subsection*{Sætninger}
		\begin{setn}
			For to grupper $(G,*)$ og $(H, \diamond)$ med neutrale elementer $e_G$ og $e_H$
			og en homomorfi $\varphi: G \too H$ da er $\varphi(e_G) = e_H$.
		\end{setn}
		\begin{proof}
			$$\varphi(e_G) = \varphi(e_G)\diamond e_H
			= \varphi(e_G)\diamond\varphi(e_G)\diamond\varphi(e_G)\inv
			= \varphi(e_G*e_G)\diamond\varphi(e_G)\inv
			= \varphi(e_G)\diamond\varphi(e_G)\inv = e_H$$
		\end{proof}
		\subsection*{Vis selv}
		\begin{setn}
			For to grupper $G$ og $H$, en homomorfi $\varphi: G \too H$
			og et element $a \in G$ da er $\varphi(a)\inv = \varphi(a\inv)$.
		\end{setn}
		\begin{setn}
			For to grupper $G$ og $H$ eksisterer der altid en homomorfi mellem dem.
		\end{setn}
		At to grupper er isomorfe betyder at deres struktur er meget ens og isomorfier
		fungerer næsten som en ækvivalens relation hvilket ses i følgende opgave.
		\begin{setn}
			Isomorfier opfylder kravende for en ækvivalens relation:

			$\cong$ er refleksiv altså $G \cong G$ for alle grupper $G$.

			$\cong$ er symmetrisk altså $G \cong H \biimp H \cong G$ for
			alle grupper $G$ og $H$.

			$\cong$ er transitiv altså $G \cong H \wedge H \cong K \imp G \cong K$
			for alle grupper $G$, $H$ og $K$.

		\end{setn}
		Årsagen til at det ikke er en ækvivalens relation skyldes at mængdelære ikke
		kan lide at konstruere en mængde af alle grupper og der dermed ikke er en mængde,
		som ækvivalens relationen kan være over.
		\begin{setn}[$\star$] \label{KerInj}
			Lad $G$ og $H$ være to grupper med identiteter $e_G$ og $e_H$
			og $\varphi: G \too H$ være en homomorfi.

			Da er $\varphi$ injektiv hvis og kun hvis $\ker(\varphi) = \{e_G\}$.
		\end{setn}
		Denne sætninger ret relevant så jeg skriver beviset i næste opdatering, det er
		dog stadig en ret god øvelse at vise.
	\newpage
	\section*{Undergrupper 23/2}
		\subsection*{Opsamling}
		Her er beviset for sætning \ref{KerInj}.
		\begin{proof}
			Lad $a,b \in G$.
			Hvis $\ker(\varphi) = {e_G}$ da ses det at
			$$\varphi(a) = \varphi(b) \imp \varphi(a*b\inv)=\varphi(a)\varphi(b)\inv=e_H
			\imp a*b\inv \in \ker(\varphi) \imp a*b\inv = e_G \imp a = b$$
			Hvis $\ker(\varphi)$ ikke er triviel er funktionen åbenlyst ikke injektiv.
		\end{proof}
		\subsection*{Definitioner}
		Fra nu af vil noten skifte over til multiplikativ notation så operationer
		er underforstået i forhold til hvor de sker, $1$ bliver brugt som enhed 
  		og der skrives $a\cdot b$ eller bare $ab$.
		\begin{defi}
			Lad $G$ være en gruppe og $H \ne Ø \subseteq G$ være en delmængde.
			Da er $H$ en undergruppe af $G$ noteret $H \le G$ hvis

			(1) $x \in H \imp x\inv \in H$

			(2) $x,y \in H \imp xy \in H$
		\end{defi}
		\begin{nota}
			Lad $G$ være en gruppe, $H \le G$ og $g \in G$. Da er der følgende notation

			$gH = \{gh | \forall h \in H\}$ Kaldet en venstresideklasse.

			$Hg = \{hg | \forall h \in H\}$ Kaldet en højresideklasse.

			$gHg\inv = \{ghg\inv | \forall h \in H\}$
			Kaldet $H$ konjugeret med $g$ ligesom $ghg\inv$ er $h$ konjugeret med $g$.
		\end{nota}
		\subsection*{Sætninger}
		\begin{setn} \label{SidKlaPar}
			Lad $G$ være en gruppe og $H \le G$. For elementer $a, b \in G$ da er
			$aH = bH$ eller $aH\cap bH = Ø$.
		\end{setn}
		\begin{proof}
			Antag at der eksisterer $c \in aH\cap bH$. Da ligger $c$ både i $aH$ og i $bH$, så
			der må eksistere $h_1, h_2$ så $c = ah_1$ og $c = bh_2$. Da ses det at
			$$ah_1 = bh_2 \imp a=bh_2h_1\inv$$
			Lad nu $d$ være et element
			i $aH$. Da ses det at
			$$d = ah_3=bh_2h_1\inv h_3$$
			Men da $h_1$, $h_2$ og $h_3$ ligger i $H$ må $h_2h_1\inv h_3$ ligge i $H$
			da $H$ er en undegruppe. Altså må $d$ ligge i $bH$ og dermed er $aH \subseteq bH$.
			Det ses symmetrisk at $bH \subseteq aH$ hvilket medfører $aH = bH$.
		\end{proof}
		\subsection*{Vis selv}
		\begin{setn}
			Lad $G$ være en gruppe og $H \le G$. Da gælder det at $1 \in H$.
		\end{setn}
		\begin{setn}
			Lad $G$ og $H$ være to grupper og $\varphi: G \too H$ være en homomorfi.
			Da er $\ker(\varphi) \le G$ og $\varphi(G) \le H$.
			
			(Note: $\varphi(G)$ er billedet af $\varphi$ ofte skrevet $\text{im}(\varphi)$.)
		\end{setn}
		\begin{setn} \label{SidKlaStø}
			Lad $G$ være en gruppe, $H \le G$ og $a,b \in G$. Da er $|aH| = |bH|$.
			Hvilket er ækvivalent med at der eksisterer en bijektion mellem $aH$ og $bH$.
		\end{setn}
		\begin{setn}[Lagrange $\star$] \label{Lagrange}
			Lad $G$ være en endelig gruppe og $H \le G$. Da gælder det at
			$$|H| \,\Big|\, |G|$$
			Læses $|H|$ deler $|G|$.
		\end{setn}
		(Hint: Benyt sætning \ref{SidKlaStø} og \ref{SidKlaPar}.)
	\newpage
	\section*{Normale undergrupper 24/2}
		Idag bliver lidt kortere.
		\subsection*{Opsamling}
		Her er beviset for \ref{Lagrange}.
		\begin{proof}
			For et givent element $g \in G$ må $g \in gH$ da $1 \in H$.
			Fra sætning \ref{SidKlaPar} ses det så at $H$ sideklasserne
			er en partition af $G$. Lad $K$ betgne mægnden af $H$ sideklasser da må
			$$|G| = \sum_{S \in K} |S|$$
			Fra \ref{SidKlaStø} fåes det at alle $H$ sideklasser har samme størelser og
			da $H$ er en $H$ sideklasse har de alle størelse $|H|$. Det ses så at
			$$|G| = \sum_{S \in K} |S| = \sum_{S \in K} |H| = |H|\cdot |K|$$
			Da $G$ er endelig må både $H$ og $K$ være endelige og $|G|$, $|H|$ og $|K|$
			må da være hele tal og derfor må $|H| \,\Big|\, |G|$.
		\end{proof}
		\subsection*{Definitioner}
		Det giver nu mening at tale om mængden af sideklasser.
		\begin{defi}
			Lad $G$ være en gruppe og $H \le G$. Da betegner $|G:H|$
			antallet af $H$ sideklasser. $|G:H|$ kan godt være uendelig.
		\end{defi}
		\begin{defi}
			Lad $G$ være en gruppe og $H \le G$. Da er $H$ en normal undergruppe hvis
			$$\forall g \in G: gHg\inv = H$$
			Hvilket skrives $H \unlhd G$.
		\end{defi}
		\subsection*{Vis selv}
		\begin{setn}
			Lad $G$ og $H$ være grupper og $\varphi: G \too H$ være en homomorfi.
			Da er $\ker(\varphi) \unlhd H$.
		\end{setn}
	\newpage
	\section*{Kvotientgrupper 25/2}
		\subsection*{Definitioner}
		\begin{defi}
			Lad $G$ være en gruppe og $H \unlhd G$ da betegner $G/H$ gruppen
			af $H$ sideklasserne hvor for to side klasser $aH$ og $bH$ hvor $a$ og $b$
			er to vilkårlige representanter er $aH\cdot bH = abH$. Denne gruppe er kaldet
			en kvotientgruppe.
			homomorfien $\pi: G \too G/H$ defineret ved $\pi(g) = gH$ er kaldet
			den kanoniske afbildning.
		\end{defi}
		\begin{defi}
			Lad $G$ være en gruppe og $H \le G$. Da er der følgende undergrupper:
			$$N_G(H) = \{g \in G| gHg\inv = H\}$$
			Kaldet normalisatoren til $H$.
			$$C_G(H) = \{g \in G| \forall h \in H: gh = hg\}$$
			Kaldet centralisatoren til $H$.
			$Z(G) = C_G(G)$ kaldet centeret i $G$.
		\end{defi}
		\subsection*{Sætninger}
		\begin{setn}
			Kvotientgruppe operationen er veldefineret.
		\end{setn}
		\begin{proof}
			Lad $a,a',b,b' \in G$ så $aH = a'H$ og $bH = b'H$. Da eksistere $h_1, h_2 \in H$
			så $a' = ah_1$ og $b' = bh_2$. Betragt nu
			$$a'b'H = ah_1bh_2H = abb\inv h_1bH$$
			Men da $H$ er en normal under gruppe ligger $b\inv h_1 b = h_3 \in H$.
			$$a'b'H = abb\inv h_1bH = abh_3H = abH$$
			Ergo er valget af representatner for gruppe operationen ligegyldig og operationen
			er dermed veldefineret.
		\end{proof}
		\begin{setn}
			Kvotientgrupper er grupper.
		\end{setn}
		\begin{proof}
			Det ses at der både er inverser, et neutralt element og at operationen er asosiativ:

			$1H\cdot aH = aH$,

			$aH\cdot a\inv H = 1H$,

			$aH(bH\cdot cH) = aH \cdot bcH = abcH = abH \cdot cH = (aH\cdot bH)cH$.
		\end{proof}
		\begin{setn} \label{SidKlaKrav}
			Lad $G$ være en gruppe og $H \le G$. Da gælder det at 
			$$a\inv b \in H \biimp aH = bH$$
		\end{setn}
		\begin{proof}
			$$aH = bH \biimp \exists h \in H: b = ah \biimp \exists h \in H: a\inv b = h \biimp
			a\inv b \in H$$
			Første biimplikation fåes fra sætning \ref{SidKlaPar}.
		\end{proof}
		\subsection*{Vis selv}
		\begin{setn}
			Lad $G$ væren en gruppe og $H \le G$, da er både $C_G(H)$ og $N_G(H)$ undergrupper
			af $G$.
		\end{setn}
		\begin{setn}
			Lad $G$ være en gruppe og $H \unlhd G$ da er
			den kanoniske afbildning en homomorfi.
		\end{setn}
		\begin{opg}
			Vis at $(3\Z,+)$ er en gruppe og bestem $\Z/3\Z$.

			Bestem $\Z/n\Z$ for et naturligt tal $n$.

			(Bemærkning: $n\Z = \{n\cdot a| \forall a \in \Z\}$.)
		\end{opg}
		\begin{setn}[Første isomorfi sætning $\star$] \label{Iso1}
			Lad $G$ og $H$ være grupper og lad $\varphi: G \too H$ være en gruppe homomorfi.
			Da er $G/\ker(\varphi) \cong \varphi(G)$.
		\end{setn}
	\newpage
	\section*{Isomorfi sætninger del 1 26/2}
		\subsection*{Opsamling}
		Her er beviset for sætning \ref{Iso1}.
		\begin{proof}
			Betragt functionen $\pi: G/\ker(\varphi) \too \phi(G)$ defineret ved
			$$\pi(g\ker(\varphi)) = \varphi(g)$$
			Først ses det at $\pi$ er veldefineret. Så antag $g\ker(\varphi) = g'\ker(\varphi)$.
			Der eksistere $k \in \ker(\varphi)$ så
			$$\pi(g\ker(\varphi)) = \varphi(g) = \varphi(g'k) = \varphi(g')\varphi(k)=\varphi(g') =
			\pi(g'\ker(\varphi))$$
			Det ses nu at $\pi$ er en homomorfi:
			$$\pi(a\ker(\varphi)b\ker(\varphi)) = \pi(ab\ker(\varphi)) = \varphi(ab) =
			\varphi(a)\varphi(b)=\pi(a\ker(\varphi))\pi(b\ker(\varphi))$$
			Det ses let at $\pi$ er surjektiv da hvis $h \in \varphi(G)$ eksistere $g$
			så $\varphi(g) = h$ og da er $\pi(g\ker(\varphi)) = \varphi(g) = h$.
			Det ses også at $\pi$ er injektiv da
			$$\pi(g\ker(\varphi)) = 1 \imp \varphi(g) = 1 \imp g \in \ker(\varphi) \imp
			g\ker(\varphi) = 1\ker(\varphi)$$
			Da er kernen af $\pi$ triviel og fra sætning \ref{KerInj} må $\pi$ være injektiv.
			Ergo er $\pi$ en isomorfi.
		\end{proof}
		\subsection*{Definitioner}
		\begin{defi}
			Lad $G$ være en gruppe og $A$ og $B$ være undergrupper af $G$. Da betegner
			$$AB = \{ab | \forall a \in A \forall b \in B\}$$
		\end{defi}
		\subsection*{Sætninger}
		\begin{setn}
			Lad $G$ være en gruppe og $A \subseteq N_G(B)$ og $B$ være undergrupper af $G$.
			Da er $AB \le G$.
		\end{setn}
		\begin{proof}
			Betragt $a \in A$ og $b \in B$ da $A \subseteq N_G(B)$ er $ab\inv a\inv = b' \in B$ og
			$$(ab)\inv = b\inv a\inv = a\inv a b \inv a \inv = a \inv b'$$
			Da $a\inv \in A$ og $b' \in B$ ses det så at $(ab)\inv \in AB$. Betragt nu
			$a_1, a_2 \in A$ og $b_1, b_2 \ in B$ da må $a_2\inv b_1 a_2 = b' \in B$ det ses så at
			$$a_1b_1a_2b_2 = a_1a_2a_2\inv b_1a_2b_2 = a_1a_2b'b_2$$
			ergo ligger $(a_1b_1)(a_2b_2)$ i $AB$ og $AB$ må være en undergruppe.
		\end{proof}
		\begin{setn}[Den anden isomorfi sætning]
			Lad $G$ være en gruppe og $A \subset N_G(B)$ og $B$ være undergrupper af $G$.
			Da er $B \unlhd AB$, $a\cap B \unlhd A$ og $AB/B \cong A/A\cap B$.
		\end{setn}
		\begin{proof}
			Beviset for at $B \unlhd AB$ og $A\cap B \unlhd A$ undlades til læseren(Dagens opgaver).
			Bestem $\varphi: A \too AB/B$ ved $\varphi(a) = aB$. Da ses det at $\varphi$ er en
			homomorfi:
			$$\varphi(aa') = aa'B = aBa'B = \varphi(a)\varphi(a')$$
			Vi bestemmer nu $\ker(\varphi)$ så betragt $a \in A$ så $\varphi(a) = 1B$
			Da må $aB = \varphi(a) = 1B$ ergo må $a \in B$ og kernen er dermed $A\cap B$.
			Det ses også at $\varphi$ er surjectiv da $aB = \varphi(a')$ for enhver representant
			$a' \in aB$. Fra sætning \ref{Iso1} fåes det så at
			$$A/A\cap B = A/\ker(\varphi) \cong \varphi(A) = AB/B$$
		\end{proof}
	\newpage
	\section*{Isomorfi sætninger del 2 27/2}
		\subsection*{Vis selv}
		\begin{setn}
			Lad $G$ være en gruppe, $H \unlhd G$, $K \unlhd G$ og $H \le K$.

			Da er $K/H \unlhd G/H$ og
		\end{setn}
		\subsection*{Sætninger}
		\begin{setn}[Tredje isomorfi sætning] \label{Iso3}
			Lad $G$ være en gruppe, $H \unlhd G$, $K \unlhd G$ og $H \le K$.

			Da er $(G/H)/(K/H) \cong G/K$
		\end{setn}
		\begin{proof}
			Definer $\varphi: G/H \too G/K$ ved $\varphi(gH) = gK$. Først ses
			det at $\varphi$ er veldefineret. Lad $gH = g'H$ da eksister $h$ så
			$g' = gh$. Da $H \le K$ ses det at
			$$\varphi(g'H) = g'K = ghK = gK$$
			Vi bestemmer nu $\ker(\varphi)$.
			\begin{align*}
				\ker(\varphi) &= \{gH \in G/H| \varphi(gH) = 1K\} \\
				&= \{gH \in G/H| gK = 1K\} \\
				&= \{gH \in G/H| g \in K\} \\
				&= K/H
			\end{align*}
			Det ses let at $\varphi$ er surjektiv og fra sætning \ref{Iso1} ses det så at
			$$(G/H)/(K/H) = (G/K)/\ker(\varphi) \cong = \varphi(G/K) = G/K$$
		\end{proof}
		Af isomorfi sætninger er det den første der er den stærkeste hvilket de to andre viser
		da beviserne falder ud ved at betragte den mest trivielle homomorfi og der efter bruge
		første isomorfi sætning.

		Næste gang begynder vi på det sidste store resultat i gruppe teori vi vil se på før vi
		går videre til ringe og legmer.
	\newpage
	\section*{Start på Sylow}
		Vi kan nu begynd at gennemgå nogle sætninger der skal bruges til at vise
		Sylows sætninger.
		\subsection*{Definitioner}
		\begin{defi}
			Lad $G$ være en gruppe og lad $g \in G$.
			Da betegner $g^0 :=1$.
			For $n \in \N$ betegner $g^n := g^{n-1}\cdot g$ og $g^{-n} := g^{-n+1}\cdot g\inv$.
		\end{defi}
		\begin{defi}
			Lad $G$ være en gruppe og lad $g \in G$.
			Da er ordenen af $g$ det mindste tal $n \in \N$ så $g^n = 1$.
			Dette skrives $|g| = n$ hvis sådan tal ikke eksistere skrives $|g| = \infty$.
		\end{defi}
		\begin{defi}
			Lad $G$ være en gruppe og $g_1, g_2, \dots, g_n \in G$. Da betegner
			$<g_1,g_2,\dots g_n>$ den mindste undergruppe af $G$ eder indeholder
			$g_1, g_2, \dots, g_n$.
		\end{defi}
		\begin{defi}
			Lad $G$ være en gruppe og $H,K \le G$. Da betegner
			$$HK := \{hk | \forall h \in H \forall k \in K\} = \bigcup_{h\in H} hK$$
		\end{defi}
		\subsection*{Vis selv}
		\begin{setn}
			Lad $G$ være en gruppe og $g \in G$. Da er $|g| = |<g>|$.
		\end{setn}
		\begin{setn}
			Lad $G$ være en gruppe, $g \in G$ og $n \in \N$. Da er
			$$|x^n| = \frac{|x|}{\gcd(|x|,n)}$$
		\end{setn}
		\subsection*{Sætninger}
		\begin{setn} \label{HK}
			Lad $G$ være en gruppe og $H, K \le G$. Da er
			$$|HK| = \frac{|H||K|}{|H \cap K|}$$
		\end{setn}
		\begin{proof}
			Da
			$$HK = \{hk | \forall h \in H \forall k \in K\} = \bigcup_{h\in H} hK$$
			Hvilket betyder at der skal tælles forskellige $K$ side klasser i $H$.
			Det ses dog for $h_1,h_2 \in H$ at
			$$h_1K = h_2K \biimp h_2\inv h_1 \in K \biimp h_2\inv h_1 \in H \cap K
			\biimp h_1(H\cap K) = h_2(H\cap K)$$
			Altså er antallet af $hK$ sideklasser for $h \in H$ antallet
			af $H\cap K$ sideklasser i $H$. Det ses fra lagrange at dette antal er
			$\frac{|H|}{|H\cap K|}$ og da hver $K$ sideklasse indeholder $|K|$ elementer
			fåes det ønskede:
			$$|HK| = \frac{|H||K|}{|H \cap K|}$$
		\end{proof}
		\begin{setn} \label{AbelOrdP}
			Lad $G$ være en abelsk gruppe hvor $p \; \Big| \; |G|$ hvor $p$ er et primtal.
			Da eksistere et element med orden $p$.
		\end{setn}
		\begin{proof}
			Sætningen gælder trivielt hvis $|G| = p$.
			Antag for induction at sætningen gælder for alle grupper $|H| < |G|$.
			Betragt et element $x$ hvor $p \; \Big | \; |x| = p\cdot n$. Da må 
			$|x^n| = p$.
			Hvis $p \; \nmid \; |x|$. Da $F$ er abelsk er
			$<x> \unlhd G$ og dermed er $|G/<x>| < |G|$. Fra Lagrange og $p \; \nmid \; |x|$
			ses det at
			$$p \; \Big | \; |G/<x>| = \frac{|G|}{|<x>|}$$
			Da eksistere en et element $\bar{y} \in G/<x>$ med $|\bar{y}| = p$. Da eksistere
			$y \in G$ så $\bar{y} = yN$. Da ses det at $y \notin <x>$ og $y^p \in <x>$. Da er
			$<y^p> \le <x>$ ergo er $<y> \ne <y^p>$ og dermed $p \; \Big | \; |y|$.
			Da er $|y| = p \cdot n$ og $|y^n| = p$.
		\end{proof}
	\newpage
	\section*{Konjugens klasser}
		Der er stadig en bid vej til Sylows sætning så vi begynder nu at se på
		konjugens klasser for at vise klasse sætningen.
		\subsection*{Definitioner}
		\begin{defi}
			Lad $G$ være en gruppe og $a,b \in G$. Da kaldes $a$ og $b$ konjugerede af
			hindanden hvis der eksistere $g \in G$ så $gag\inv = b$.
		\end{defi}
		\begin{defi}
			Lad $G$ være en gruppe og $H \le G$ da kaldes $H$
			en konjugens klasse hvis der eksistere et element $h \in G$ så
			$$H = \{ghg\inv | \forall g \in G\}$$
		\end{defi}
		\subsection*{Sætninger}
		\begin{setn}
			Lad $G$ være en gruppe og lad $H$ være en konjugens klasse med
				$$H = \{ghg\inv | \forall g \in G\}$$
			For et element $h$. Da er
			$$|H| = |G:C_G(h)|$$
		\end{setn}
		\begin{proof}
			Betragt en afbildningen $f$ fra sideklasserne til $C_G(h)$ til $H$ med
			$$f(aC_G(h)) = aha\inv$$
			Først skal det lige vises at $f$ er veldefineret.
			Fra sætning \ref{SidKlaKrav} fåes det at
			$$aC_G(h)=bC_G(h) \biimp b\inv a \in C_G(h) \biimp h = (b\inv a)h(b\inv a)\inv =
			b\inv aha\inv b \biimp aha\inv = bhb\inv \biimp f(a) = f(b)$$
			Dette viser også at $f$ også er injektiv. Så mangler det blot at $f$ er surjektiv.
			Betragt $a \in H$ da eksitere $g$ så
			$$a = ghg\inv = f(gC_G(H))$$
			Hvilket viser at $f$ er bijektiv hvilket medfører at
			$$|H| = |G:C_G(h)|$$
		\end{proof}
		\begin{setn}[Klasse sætningen]
			Lad $G$ være en endelig gruppe og lad $g_1,g_2,\dots,g_r$ være representanter
			for alle konjugensklasser med mere end et element. Da er
			$$|G| = |Z(G)| + \sum_{i=1}^r|G:C_G(g_i)|$$
		\end{setn}
		\begin{proof}
			Fra sætning \ref{KonjPar} ses
			det at konjugens klasser er en parition af $G$ så alle elementer ikke betragtet
			i de førnævnte konjugens klasser er de elementer hvor
			$$\{ghg\inv | \forall g \in G\} = \{a\}$$
			Hvilket netop er alle elementer i $Z(G)$. Dette giver
			$$|G| = |Z(G)| + \sum_{i=1}^r|\{ag_ia\inv | \forall a \in G\}| =
			|Z(G)| + \sum_{i=1}^r|G:C_G(g_i)|$$
		\end{proof}
		\subsection*{Vis selv}
		\begin{setn} \label{KonjPar}
			Lad $G$ være en gruppe og $a,b \in G$ være konjugere af hindanden.
			Da er
			$$\{gag\inv | \forall g \in G\} = \{gbg\inv | \forall g \in G\}$$
			og
			$$a \in \{gag\inv | \forall g \in G\}$$
		\end{setn}
	\newpage
	\section*{Eksistens af Sylow gruppe}
		Vi er nu klar til at definere og vise eksistensen af Sylow grupper
		\subsection*{Definitioner}
		\begin{defi}
			Lad $G$ være en endelig gruppe, $p$ være et primtal hvor $\exists \alpha,k \in \N_0$ så
			$|G| = p^\alpha k$ med $p \nmid k$. Da er en Sylow $p$-undergruppe en gruppe med
			orden $p^\alpha$.
		\end{defi}
		\subsection*{Vis selv}
		\begin{setn} \label{UndGrupOp}
			Lad $G$ være en gruppe, $H \unlhd G$ og $\bar{P} \le G/H$.
			Da er
			$$P = \bigcup_{S \in \bar{P}}S$$
			en undergruppe af $G$ med
			$|P| = |\bar{P}|\cdot|H|$.
		\end{setn}
		\begin{setn}
			Lad $G$ være en gruppe og $H \le Z(G)$. Da er $H \unlhd G$.
		\end{setn}
		\subsection*{Sætninger}
		\begin{setn}[Sylows sætning del 1]
			Lad $G$ være en endelig gruppe og $p$ være et primtal.
			Da eksistere en Sylow $p$-undergruppe i $G$.
		\end{setn}
		\begin{proof}
			Vi vil vise sætningen ved induktion. Sætningen gælder trivielt for $|G|=1$.
			Da $G \le G$ er en undergruppe med $|G| = p^0$.
			Så antag at alle grupper med orden mindre end $|G|$ har Sylow $p$-undergrupper.
			Antag at $p \; \Big | \; |Z(G)|$. Fra sætning \ref{AbelOrdP} eksistere et element
			$g \in Z(G)$ med orden $p$ da $Z(G)$ er abelsk. Da er $N = <g>$ en undergruppe med
			orden $p$. Da $N \le Z(G)$ er $G/N$ veldefineret.
			Da $N \ne \{1\}$ er $|G/N| < |G|$. Specielt er $|G| = \frac{|G|}{p} = p^{\alpha-1}k$.
			Fra inductions antagelsen eksistrere en Sylow $p$-undergruppe $\bar{P}$ af $G/N$
			med orden $p^{\alpha-1}$. Fra sætning \ref{UndGrupOp} fåes en undergruppe $P$
			med $|P| = |\bar{P}|\cdot|N| = p^\alpha$. Hvilket fuldføre caset $p \;\Big |\;|Z(G)|$.
			Hvi dette ikke gælder kan klasse sætningen betragtes
			$$|G| = |Z(G)| + \sum_{i=1}^r|G:C_G(g_i)|$$
			Da $p$ ikke deler $|Z(G)|$. Må der eksistere et $g_i$ så $p \nmid \; |G:C_G(g_i)|$
			Men da $|G:C_G(g_i)| = \frac{G}{|C_G(g_i)|}$ må $p^\alpha$ dele $|C_G(g_i)|$.
			Da $g_i$ per definition ikke ligger i $Z(G)$ er $C_G(g_i) < G$ og dermed
			er $|C_G(g_i)| < |G|$. Fra induktions antagelsen eksistere en Sylow $p$-undergruppe
			$P$ af $C_G(g_i)$ med orden $p^\alpha$ men da er $P \le C_G(g_i) \le G$.
		\end{proof}
	\newpage
	\section*{Orbits under konjugation}
		\subsection*{Definitioner}
		\begin{defi}
			Lad $G$ være en gruppe og $H,K \le G$. Betragt mængden
			$$S = \{gHg\inv| \forall g \in G\}$$
			For et element $A \in S$ kaldes følgende mængde for
			$A$'s orbit under operation fra $K$.
			$$O_A = \{kAk\inv| \forall k \in K\}$$
		\end{defi}
		\subsection*{Vis selv}
		\begin{setn}
			Lad $G$ være en gruppe og $H,K \le G$. Da er $H\cup K \le H$ og $H\cup N \le N$.
		\end{setn}
		\begin{setn} \label{KonjGrup}
			Lad $G$ være en endelig gruppe,$g \in G$ og $H \le G$. Da er $gHg\inv \le G$.
		\end{setn}
		\begin{setn}
			Lad $G$ være en gruppe, $H \le G$ og $a,b \in G$ da gælder det at
			$$H=(a\inv b)H(b\inv a) \biimp aHa\inv = bHb\inv$$
		\end{setn}
		\begin{setn}
			Lad $G$ være en gruppe, $H,K \le G$, $S = \{gHg\inv| \forall g \in G\}$
			og $O$ være mængden af alle orbits af elementer i $S$ under operation fra $K$.
			Skrevet som mængde:
			$$O = \{\{kAk\inv| \forall k \in K\}| \forall A \in S\}$$
			Da er $O$ en partition af $S$.
		\end{setn}
		\subsection*{Sætninger}
		\begin{setn} \label{OrbStøL}
			Lad $G$ være en gruppe, $H,K \le G$, $S = \{kHk\inv| \forall k \in K\}$. Da er
			$$|S| = |K:N_K(H)|$$
		\end{setn}
		\begin{proof}
			Lad $M$ være mængden af $N_K(A)$ sideklasser i $K$.
			Betragt $f:M \too S$ defineret ved $f(aN_K(A)) = aAa\inv$.
			Først ses det at $f$ er veldefineret og injektiv. Betragt $a,b \in K$. Da er
			\begin{align*}
				aN_K(A) = bN_K(A) &\biimp a\inv b \in N_K(A) \\
				&\biimp A = (a\inv b)A(a\inv b)\inv = (a\inv b)A(b\inv a) \\
				&\biimp f(aN_K(A)) = aAa\inv = bAb\inv = f(bN_K(A))
			\end{align*}
			Det ses let at $f$ er surjektiv da for alle $kAk\inv$ er $f(kN_K(A)) = kAk\inv$.
			Ergo er $f$ en bijektion hvilket viser det ønskede.
		\end{proof}
		\begin{koro} \label{ConjTal}
			Lad $G$ være en gruppe, $H \le G$ og $S = \{gHg\inv| \forall g \in G\}$, $A \in S$.
			Da er
			$$|S| = |G:N_G(H)|$$
		\end{koro}
		\begin{proof}
			Dette følger fra sætning \ref{OrbStøL} med $K = G$.
		\end{proof}
		\begin{koro} \label{OrbStø}
			Lad $G$ være en gruppe, $H,K \le G$, $S = \{gHg\inv| \forall g \in G\}$, $A \in S$
			og $O_A$ være $A$'s orbit under operation fra $K$. Da er
			$$|O_A| = |K:N_K(A)|$$
		\end{koro}
		\begin{proof}
			Dette følger fra sætning \ref{OrbStøL} da $A \le G$ fra sætning \ref{KonjGrup}.
		\end{proof}
		\newpage
	\section*{Sylows sætninger}
		\subsection*{Definitioner}
		\begin{defi}
			Lad $G$ være en gruppe og $p$ være et primtal. En $p$-undegruppe er en undergruppe
			med størelse $p^a$ for et $a \in \N$.
		\end{defi}
		\begin{defi}
			Lad $G$ være en endelig gruppe og $p$ være et primtal. Da betegner $Syl_p(G)$ mængden
			af Sylow $p$-undergrupper i $G$ og $n_p(G)$ defineres til at være $|Syl_p(G)|$.
		\end{defi}
		\subsection*{Vis selv}
		\begin{setn}
			Lad $G$ være en gruppe, $H \le G$ og $g \in G$. Da er $H \cong gHg\inv$.
		\end{setn}
		\begin{setn} \label{HKGrup}
			Lad $G$ være en gruppe og $H,K \le G$ med $H \le N_G(K)$. Da er $HK \le G$ med
			$H \le HK$ og $K \le HK$.
		\end{setn}
		\subsection*{Sætninger}
		\begin{setn} \label{SylNormCap}
			Lad $G$ være en endelig gruppe, $p$ være et primtal, $P \in Syl_p(G)$
			og $Q \le G$ være en $p$-undergruppe. Da er
			$$Q\cap N_G(P) = Q\cap P$$
		\end{setn}
		\begin{proof}
			Lad $H = Q\cap N_G(P)$.
			Da $P \le N_g(P)$ må $Q\cap P \le H$. Per definition er $Q \le H$.
			Vi mangler derfor kun at vise at $H \le P$. Fra sætning \ref{HK} at
			$$|HP| = \frac{|H||P|}{|H \cap P|}$$
			Da $H \cap P \le P$ må $H \cap P$ være en $p$-undergruppe. Tilsvarende
			er $H \le Q$ hvilket medfører at $H$ er en $p$-undergruppe.
			Da må $|H|$,$|P|$ pg $|H \cap P|$ være potenser af $p$ og dermed er
			$HP$ en $p$-undergruppe. Da $H \le N_p(P)$ ses det fra sætning \ref{HKGrup}
			at $P \le HP$. Dog er $P$ en maksimal $p$-undergruppe og derfor må
			$P = HP$. Da er $H \le HP = P$ hvilket viser det ønskede.
		\end{proof}
		\begin{setn} \label{Syl2}
			Lad $G$ være en endelig gruppe, $p$ være et primtal, $P \in Syl_p(G)$ og $Q \le G$
			være en $p$-undergruppe. Da eksistere $g \in G$ så $Q \le gPg\inv$.
		\end{setn}
		\begin{proof}
			Betragt $S = \{gPg\inv| \forall g \in P\}$ med $S = \{P_1,P_2,\dots,P_r\}$
			og orbitsne i $S$ under konjugation fra $Q$ $O = \{O_1,O_2,\dots,O_s\}$.
			Da er $|O_1|+|O_2|+\dots+|O_s| = r$. Da kan $P_1,P_2,\dots,P_r$ omnavngives
			så $P_i \in O_i$ for $0<i\le s$. Da $P_i \cong P$ er en $p$-sylowgruppes ses
			det fra sætning \ref{OrbStø} og \ref{SylNormCap} at
			$|O_i| = |Q: N_Q(P_i)| = |Q: N_G(P_i)\cap Q| = |Q:P_i\cap Q|$.
			Specielt gælder dette for valget $Q = P_1$. For $i \ne 1$ er
			$P_i \ne P_1$ og dermed er $|P_1:P_i\cap p_1| > 1$ og må dele $|P_1|$
			som er en potens af $p$. Ergo må $p \big | \; |O_i|$ for $i \ne 1$.
			Det ses også at $|O_1| = |P_1:P_1\cap p_1| = 1$
			Da er $r = |O_1|+|O_2|+\dots+|O_s| = 1 + |O_2|+\dots+|O_s|$
			og dermed er $r \cong 1 \; (\mathrm{mod}\; p)$.

			Antag nu for modstrid at der ikke eksistere $i$ så $Q \subseteq P_i$.
			Da må $|Q:P_i\cap Q| > 1$ for alle $i$ og da $Q$ er en $p$-undergruppe må
			$p$ dele $|Q:P_i\cap Q| = |O_i|$. Da må $p$ dele $r = |O_1|+|O_2|+\dots+|O_s|$
			hvilket er i modstrid med $r \cong 1 \; (\mathrm{mod}\; p)$.
		\end{proof}
		\begin{setn} \label{Syl3}
			Lad $G$ være en gruppe og $p$ være et primtal.
			Da er $n_p(G) \cong 1 \; (\mathrm{mod}\; p)$ og $n_p(G) \big| \; |G|$.
		\end{setn}
		\begin{proof}
			Lad $P \in Syl_p(G)$ og $S = \{gPg\inv| \forall g \in P\}$. Betragt nu $Q \in Syl_p(G)$.
			Da siger sætning \ref{Syl2} at der eksister $g$ så $Q \subseteq gPg\inv$.
			Da $|Q| = |P| = |gPg\inv|$ må $Q = gPg\inv$ og dermed er $S = Syl_p(G)$.
			I beviset til sætning \ref{Syl2} blev det vist at $|S| \cong 1  \; (\mathrm{mod}\; p)$.
			Fra sætning \ref{ConjTal} ses det at $|S| = |G:N_G(H)|$ og dermed
			må $n_p(G) = |S|$ dele $|G|$.
		\end{proof}
	\ud{
		\subsection*{Opsamling}
		\subsection*{Definitioner}
		\subsection*{Sætninger}
		\subsection*{Vis selv}
	\section*{Til senere}
		\begin{defi}
			En tupel $(R,+,\cdot)$ med en mængde $R$ og binære operationer $+, \cdot$
			kaldes en ring hvis:

			(1) $(R,+)$ er en gruppe hvor neutral elementet kaldes $0$.

			(2) $\cdot$ er en acosiativ operation.

			(3) $\cdot$ distribuere over $+$ alstå $\forall a,b,c \in R: a*(b+c)=a*b+a*c$
			og $(a+b)*c = a*c+b*c$.
		\end{defi}
	\section*{Til senere}
	}

\end{document}
